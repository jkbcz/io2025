\documentclass[12pt,a4paper]{article}

% --- Pakiety dla języka polskiego i czcionek ---
\usepackage[utf8]{inputenc}
\usepackage[T1]{fontenc}
\usepackage[polish]{babel}
\usepackage{geometry}
\geometry{margin=2.5cm}
\usepackage{graphicx} % KONIECZNE DO DIAGRAMÓW
\usepackage{amsmath}  % DO STRZAŁEK I SYMBOLI
\usepackage{float}    % POMAGA W USTAWIANIU DIAGRAMÓW [H]

% --- Dane do strony tytułowej ---
\title{
    \vspace{3cm}
    \textbf{\Huge Projekt Systemu Informatycznego} \\
    \vspace{1cm}
    \Large TransTime - System dynamicznej informacji pasażerskiej i analizy rozkładów jazdy w czasie rzeczywistym \\
    \vspace{2cm}
}

\author{
    \textbf{Autorzy:} \\
    Adam Czakon \\
    Jakub Czyż \\
    Jakub Czajka
}

\date{\vfill \today}

\begin{document}

% --- STRONA TYTUŁOWA I SPISY ---
\maketitle
\thispagestyle{empty}
\newpage

\tableofcontents
\listoffigures

\newpage


% --- PUNKT 2: KONCEPCJA SYSTEMU ---
\section{Koncepcja systemu}

Celem projektu jest stworzenie systemu informatycznego umożliwiającego publikację aktualnych rozkładów jazdy komunikacji miejskiej w Internecie. W odróżnieniu od tradycyjnych, statycznych tabel, system ten oferuje dynamiczne śledzenie czasu odjazdu na podstawie rzeczywistych danych.

Główne założenia systemu to:
\begin{itemize}
    \item \textbf{Publikacja rozkładów:} Udostępnienie rozkładów jazdy dla konkretnych linii oraz poszczególnych przystanków.
    \item \textbf{Analiza ruchu (Traffic):} Uwzględnienie aktualnego stanu przejezdności dróg w mieście.
    \item \textbf{Estymacja czasu (ETA):} Obliczanie faktycznego czasu dojazdu pojazdu do przystanku z uwzględnieniem opóźnień i zdarzeń losowych.
    \item \textbf{Informacja pasażerska:} Przekazanie użytkownikowi jasnej informacji, czy dany kurs nie wypadł z rozkładu i ile minut pozostało do jego przyjazdu.
\end{itemize}

System ma za zadanie zwiększyć komfort podróżnych poprzez dostarczenie wiarygodnych danych w czasie rzeczywistym, wzorując się na nowoczesnych systemach typu ITS (Intelligent Transportation Systems).



% --- PUNKT 3: DIAGRAM PRZYPADKÓW UŻYCIA ---
\newpage
\section{Diagram przypadków użycia}

Poniższy diagram identyfikuje głównych aktorów systemu oraz kluczowe funkcjonalności (przypadki użycia) niezbędne do realizacji celów biznesowych systemu TransTime.

\begin{figure}[H]
    \centering
    \includegraphics[width=\textwidth]{DPU5.drawio.png}
    \caption{Diagram przypadków użycia systemu TransTime.}
    \label{fig:use_case}
\end{figure}

\subsection*{Opis aktorów i funkcjonalności:}
\begin{itemize}
    \item \textbf{Pasażer:} Użytkownik końcowy, który wyszukuje połączenia, przegląda rozkłady linii oraz sprawdza estymowany czas przyjazdu (ETA).
    \item \textbf{Dyspozytor:} Pracownik MPK odpowiedzialny za wprowadzanie komunikatów o zdarzeniach losowych (np. awaria tramwaju, objazd).
    \item \textbf{System GPS/Traffic:} Zewnętrzne źródło danych dostarczające w trybie ciągłym informacje o współrzędnych pojazdów i natężeniu ruchu.
    \item \textbf{Administrator Systemu: } Odpowiedzialny za zarządzanie danymi podstawowymi i aktualizację bazy rozkładów.
\end{itemize}



% --- PUNKT 4: DZIEDZINOWY DIAGRAM KLAS ---
\newpage
\section{Dziedzinowy diagram klas}

Dziedzinowy diagram klas przedstawia kluczowe pojęcia biznesowe występujące w systemie oraz relacje zachodzące między nimi. Koncentruje się on na strukturze informacji o rozkładach jazdy i danych napływających w czasie rzeczywistym.

\begin{figure}[h]
    \centering
    \includegraphics[width=1.1\textwidth]{ddk.png}
    \caption{Dziedzinowy diagram klas systemu informacji pasażerskiej.}
    \label{fig:domain_model}
\end{figure}

\subsection{Opis kluczowych klas dziedzinowych}
\begin{itemize}
    \item \textbf{Linia:} Reprezentuje numer linii (np. „Tramwaj 8”) oraz typ pojazdu. Grupuje kursy realizowane według rozkładu.
    \item \textbf{Kurs (Trip):} Konkretna realizacja linii o określonej godzinie, powiązana z pojazdem. Zawiera status i umożliwia obliczanie ETA.
    \item \textbf{Postój:} Reprezentuje postoje kursu na przystankach, zawiera informacje o kolejności i czasach planowanych oraz rzeczywistych.
    \item \textbf{Przystanek:} Punkt na mapie z nazwą i lokalizacją GPS, obsługiwany przez wiele linii i kursów.
    \item \textbf{Pojazd:} Fizyczna jednostka realizująca kurs, przesyłająca swoje położenie i prędkość.
    \item \textbf{Zdarzenie drogowe:} Informacja o utrudnieniach na drodze, mająca określony czas i obszar działania. Wpływa na obliczenia czasu przejazdu, ale nie jest trwale powiązana z kursem.
\end{itemize}

\subsection{Relacje}
Zastosowano następujące powiązania:
\begin{itemize}
    \item Jedna \textbf{Linia} posiada wiele \textbf{Kursów}.
    \item Każdy \textbf{Kurs} składa się z wielu \textbf{Postojów} na \textbf{Przystankach}, ułożonych w określonej kolejności.
    \item \textbf{Pojazd} jest przypisany do dokładnie jednego \textbf{Kursu} w danym momencie.
    \item \textbf{Zdarzenia drogowe} mają przypisany obszar działania i wpływają dynamicznie na obliczanie czasu przejazdu kursów, lecz nie są z nimi trwale powiązane.
\end{itemize}


% --- PUNKT 5: SCENARIUSZE PRZYPADKÓW UŻYCIA ---
\newpage
\section{Scenariusze przypadków użycia}

Poniższe tabele szczegółowo opisują przebieg interakcji użytkownika z systemem dla wybranych, kluczowych przypadków użycia.

\subsection{Scenariusz 1: Sprawdzenie rzeczywistego czasu przyjazdu (ETA)}

\begin{table}[h]
    \centering
    \begin{tabular}{|l|p{10cm}|}
        \hline
        \textbf{Nazwa przypadku}     & Sprawdzenie rzeczywistego czasu przyjazdu (Live)            \\ \hline
        \textbf{Aktor główny}        & Pasażer                                                     \\ \hline
        \textbf{Aktorzy wspierający} & System GPS, API Traffic                                     \\ \hline
        \textbf{Warunek wstępny}     & Użytkownik znajduje się na ekranie wyboru przystanku.       \\ \hline
        \textbf{Przebieg główny}     &
        1. Użytkownik wybiera konkretny przystanek z listy lub mapy. \newline
        2. System pobiera listę nadchodzących kursów dla tego przystanku. \newline
        3. System odczytuje aktualną pozycję pojazdów przypisanych do tych kursów. \newline
        4. System analizuje dane o natężeniu ruchu na trasie pojazdu. \newline
        5. System oblicza estymowany czas przyjazdu (ETA). \newline
        6. System wyświetla użytkownikowi odświeżaną listę odjazdów.                               \\ \hline
        \textbf{Sytuacje wyjątkowe}  &
        1a. Brak danych GPS z pojazdu – system wyświetla czas teoretyczny (z rozkładu) z adnotacją „brak danych live”. \newline
        2a. Kurs został anulowany przez dyspozytora – system wyświetla status „Wypadł z rozkładu”. \\ \hline
    \end{tabular}
\end{table}

\subsection{Scenariusz 2: Dodanie komunikatu o utrudnieniach}

\begin{table}[h]
    \centering
    \begin{tabular}{|l|p{10cm}|}
        \hline
        \textbf{Nazwa przypadku} & Zarządzanie komunikatem o utrudnieniach drogowych                                       \\ \hline
        \textbf{Aktor główny}    & Dyspozytor                                                                              \\ \hline
        \textbf{Warunek wstępny} & Dyspozytor jest zalogowany do panelu administracyjnego.                                 \\ \hline
        \textbf{Przebieg główny} &
        1. Dyspozytor wybiera opcję „Dodaj komunikat”. \newline
        2. System wyświetla formularz wyboru linii lub obszaru miasta. \newline
        3. Dyspozytor wprowadza treść komunikatu (np. „Awaria sieci trakcyjnej”) i określa przewidywany czas trwania. \newline
        4. Dyspozytor zatwierdza komunikat. \newline
        5. System natychmiastowo publikuje informację na tablicach przystankowych online oraz w widoku mapy dla pasażerów. \\ \hline
    \end{tabular}
\end{table}

\subsection{Scenariusz 3: Wyszukiwanie rozkładu jazdy konkretnej linii}

\begin{table}[h]
    \centering
    \begin{tabular}{|l|p{10cm}|}
        \hline
        \textbf{Nazwa przypadku} & Przeglądanie  rozkładu linii                                  \\ \hline
        \textbf{Aktor główny}    & Pasażer                                                       \\ \hline
        \textbf{Przebieg główny} &
        1. Użytkownik wprowadza numer linii w wyszukiwarce. \newline
        2. System wyświetla listę kierunków (pętli docelowych) dla tej linii. \newline
        3. Użytkownik wybiera kierunek oraz konkretny przystanek z trasy. \newline
        4. System generuje tabelaryczny rozkład jazdy (godziny i minuty) na dany dzień tygodnia. \\ \hline
    \end{tabular}
\end{table}

\subsection{Scenariusz 4: Estymacja czasu dojazdu (Proces Systemowy)}

\begin{table}[h]
    \centering
    \begin{tabular}{|l|p{10cm}|}
        \hline
        \textbf{Nazwa przypadku}     & Estymacja czasu dojazdu na podstawie Traffic              \\ \hline
        \textbf{Aktor główny}        & System (Proces automatyczny)                              \\ \hline
        \textbf{Aktorzy wspierający} & System zewnętrzny (Google Maps API / Traffic API)         \\ \hline
        \textbf{Przebieg główny}     &
        1. System identyfikuje aktualną pozycję pojazdu (GPS). \newline
        2. System pobiera dane o natężeniu ruchu na odcinkach drogi między pojazdem a kolejnymi przystankami. \newline
        3. System porównuje czas przejazdu teoretyczny z czasem uwzględniającym korki. \newline
        4. System aktualizuje przewidywaną godzinę przyjazdu w bazie danych czasu rzeczywistego. \\ \hline
    \end{tabular}
\end{table}

\subsection{Scenariusz 5: Śledzenie pojazdu na mapie interaktywnej}

\begin{table}[h]
    \centering
    \begin{tabular}{|l|p{10cm}|}
        \hline
        \textbf{Nazwa przypadku}     & Wizualizacja pozycji pojazdu na mapie                  \\ \hline
        \textbf{Aktor główny}        & Pasażer                                                \\ \hline
        \textbf{Aktorzy wspierający} & Moduł map (np. OpenStreetMap)                          \\ \hline
        \textbf{Przebieg główny}     &
        1. Użytkownik wybiera opcję "Mapa Live". \newline
        2. System wyświetla mapę miasta z naniesionymi przystankami. \newline
        3. System pobiera pozycje wszystkich aktywnych pojazdów danej linii. \newline
        4. System nanosi ikony pojazdów na mapę, wskazując kierunek ich poruszania się. \newline
        5. Użytkownik klika w ikonę pojazdu, aby zobaczyć numer boczny i aktualne opóźnienie. \\ \hline
    \end{tabular}
\end{table}
\newpage
\subsection{Scenariusz 6: Aktualizacja bazy rozkładów (Import danych)}

\begin{table}[h]
    \centering
    \begin{tabular}{|l|p{10cm}|}
        \hline
        \textbf{Nazwa przypadku} & Import nowego rozkładu jazdy (GTFS) \\ \hline
        \textbf{Aktor główny}    & Administrator Systemu               \\ \hline
        \textbf{Przebieg główny} &
        1. Administrator przesyła plik z nowym harmonogramem do modułu zarządzania danymi. \newline
        2. System weryfikuje poprawność danych (spójność przystanków i linii). \newline
        3. System aktualizuje tabele rozkładów statycznych w bazie danych. \newline
        4. System generuje potwierdzenie pomyślnej aktualizacji.       \\ \hline
    \end{tabular}
\end{table}

\subsection{Scenariusz 7: Obsługa kursu wypadającego z rozkładu (Anulowanie kursu)}

\begin{table}[h]
    \centering
    \begin{tabular}{|l|p{10cm}|}
        \hline
        \textbf{Nazwa przypadku}     & Oznaczenie kursu jako anulowanego                                                                                   \\ \hline
        \textbf{Aktor główny}        & Dyspozytor                                                                                                          \\ \hline
        \textbf{Aktorzy wspierający} & System (Proces automatyczny)                                                                                        \\ \hline
        \textbf{Warunek wstępny}     & Kurs jest widoczny w systemie jako aktywny, ale wystąpiły przeszkody w jego realizacji (awaria, brak kontaktu GPS). \\ \hline
        \textbf{Przebieg główny}     &
        1. System generuje alert o braku sygnału GPS z pojazdu przez ponad 5 minut lub o zgłoszonej awarii technicznej. \newline
        2. Dyspozytor odbiera powiadomienie w panelu sterowania. \newline
        3. Dyspozytor weryfikuje status pojazdu (np. poprzez kontakt radiowy). \newline
        4. Dyspozytor wybiera opcję „Anuluj kurs” w systemie dla danej linii i godziny. \newline
        5. System aktualizuje status kursu na „Wypadł z rozkładu”. \newline
        6. Informacja zostaje natychmiast rozesłana do modułów pasażerskich (Tablice Live, Mapa).                                                          \\ \hline
        \textbf{Sytuacje wyjątkowe}  &
        4a. Dyspozytor stwierdza, że to tylko błąd GPS – nie anuluje kursu, system nadal wyświetla czas teoretyczny z adnotacją „Brak danych live”.        \\ \hline
    \end{tabular}
\end{table}

% --- PUNKT 6: SPECYFIKACJA DFD ---
\newpage
\section{Specyfikacja DFD (Data Flow Diagram)}

Specyfikacja przepływu danych obrazuje sposób, w jaki informacje są pobierane, przetwarzane i składowane w systemie TransTime.

\subsection{Diagram wstępny (Poziom 0 - Kontekstowy)}
Diagram kontekstowy przedstawia system jako jedną funkcję procesową i jego interakcje z otoczeniem.

\begin{figure}[H]
    \centering
    \includegraphics[width=1\textwidth]{dfd_0.png}
    \caption{Diagram kontekstowy (DFD Poziom 0).}
\end{figure}

\textbf{Główne przepływy danych na poziomie 0:}
\begin{itemize}
    \item \textbf{Pasażer:} Zapytanie o rozkład $\rightarrow$ System; Wyświetlenie ETA $\leftarrow$ System.
    \item \textbf{System GPS:} Aktualna pozycja pojazdu $\rightarrow$ System.
    \item \textbf{Zarząd Dróg (Traffic API):} Dane o natężeniu ruchu $\rightarrow$ System.
    \item \textbf{Dyspozytor:} Parametry linii i komunikaty $\rightarrow$ System.
\end{itemize}

\subsection{Dekompozycja procesów (Poziom 1 - systemowy)}
Na tym poziomie system został rozbity na główne procesy funkcjonalne.

\begin{figure}[H]
    \centering
    \includegraphics[width=1.1\textwidth]{dfd_1.png}
    \caption{Dekompozycja procesów (DFD Poziom 1).}
\end{figure}

\textbf{Wykaz procesów atomowych:}
\begin{enumerate}
    \item \textbf{P1. Aktualizacja rozkładów:} Przechowywanie statycznych godzin odjazdów.
    \item \textbf{P2. Estymacja czasu przyjazdu (ETA):} Proces łączący dane statyczne, pozycję GPS oraz komunikaty o utrudnieniach w celu obliczenia faktycznego opóźnienia.
    \item \textbf{P3. Wyświetleniu rozkładu:} Wyświetlenie informacji na temat kursów obsługiwanych przez daną linię.
    \item \textbf{P4. Publikacja informacji:} Generowanie widoków dla pasażera (tablica przystankowa, mapa).
    \item \textbf{P5. Obsługa alertów:} Rejestracja komunikatów o zdarzeniach drogowych.
    \item \textbf{P6. Anulowanie kursu:} Weryfikacja aktualności anulowanego kursu
\end{enumerate}

\subsection{Magazyny danych (Data Stores)}
W systemie zidentyfikowano następujące bazy danych:
\begin{itemize}
    \item \textbf{D1 Rozkłady jazd:} Dane statyczne o liniach i przystankach.
    \item \textbf{D2 Komunikaty o utrudnieniach:} Baza aktywnych utrudnień i ogłoszeń.
\end{itemize}

\subsection{Opis przepływów danych (Słownik danych)}
Poniższa tabela opisuje strukturę najważniejszych informacji krążących w systemie, widocznych na diagramach DFD.

\begin{table}[h]
    \centering
    \begin{tabular}{|l|l|p{7cm}|}
        \hline
        \textbf{Nazwa przepływu} & \textbf{Źródło / Cel}       & \textbf{Zawartość (Atrybuty)}                                              \\ \hline
        Dane GPS                 & Pojazd $\rightarrow$ P2     & ID pojazdu, długość i szerokość geogr., średnia prędkość.               \\ \hline
        Dane Traffic             & System GPS $\rightarrow$ P2 & ID odcinka drogi, współczynnik zakorkowania (0-1), średnia prędkość.       \\ \hline
        Wynik ETA                & P2 $\rightarrow$ Pasażer    & ID kursu, ID przystanku, przewidywana minuta przyjazdu, status opóźnienia. \\ \hline
        Zapytanie pasażera       & Pasażer $\rightarrow$ P2    & ID przystanku lub numer linii, koordynaty użytkownika.                     \\ \hline
        Alert drogowy            & Dyspozytor $\rightarrow$ P5 & Typ zdarzenia, opis tekstowy, dotknięty odcinek drogi, czas wygaśnięcia.      \\ \hline
    \end{tabular}
\end{table}

\subsection{Dekompozycja procesu poziomu 1-ego}
Na tym poziomie proces 2 został zdekomponowany na podprocesy.

\begin{figure}[H]
    \centering
    \includegraphics[width=1\textwidth]{dfd_2.png}
    \caption{Dekompozycja procesu 2.}
\end{figure}

\subsection{Procesy atomowe (Specyfikacja logiczna)}
Najniższy poziom dekompozycji (procesy atomowe) realizuje następującą logikę:
\begin{itemize}
    \item \textbf{P2.1 (Pobranie danych trasy):} Przekazuje miejsce docelowe oraz godzinę przyjazdu oczekiwaną przez pasażera do kolejnego procesu.
    \item \textbf{P2.2 (Szukanie najbliższych linii):} Pobiera rozkłady, które obejmują kursy z przystankami znajdującymi się w bliskiej odległości do miejsca docelowego oraz odjazdu.
    \item \textbf{P2.3 (Weryfikacja aktualności trasy):} Na bazie odnalezionych kursów i komunikatów o utrudnieniach przesiewa nieaktualne kursy.
    \item \textbf{P2.4 (Obliczanie daty przyjazdu):} Na podstawie GPS pojazdu, średnich prędkości i współrzędnych przystanku przewiduje pozostały czas do miejsca docelowego.
\end{itemize}

% --- PUNKT 7: SYSTEMOWY DIAGRAM KLAS ---
\newpage
\section{Systemowy diagram klas}

Systemowy diagram klas przedstawia techniczną strukturę systemu, definiując typy danych, metody oraz szczegółowe powiązania między komponentami oprogramowania.

\begin{figure}[H]
    \centering
    \includegraphics[width=\textwidth]{SDK.png}
    \caption{Systemowy diagram klas (widok implementacyjny).}
    \label{fig:system_classes}
\end{figure}

\subsection{Specyfikacja techniczna klas}

\begin{itemize}

    \item \textbf{KalkulatorETA}: Silnik obliczeniowy czasu przybycia pojazdu na dany przystanek.
          \begin{itemize}
              \item Atrybuty: brak stanów przechowywanych na poziomie klasy (stateless)
              \item Metody: \texttt{+ obliczETA(idPojazdu: UUID, przystanek: Przystanek): TimeDelta}
              \item Opis: Metoda wykorzystuje zewnętrzne serwisy, takie jak \texttt{SerwisPostojow} (dane o rozkładach i postojach) oraz \texttt{SerwisAdministracyjny} (utrudnienia), aby precyzyjnie obliczyć przewidywany czas przybycia.
          \end{itemize}

    \item \textbf{SerwisPostojow}: Serwis odpowiedzialny za zarządzanie danymi o postojach pojazdów i ich rozkładach.
          \begin{itemize}
              \item Metody: \texttt{+ pobierzPostoj(idPostoju: UUID): Postoj}, \texttt{+ uaktualnijPostoj(idPostoju: UUID, postoj: Postoj): void}, \texttt{+ usunPostoj(idPostoju: UUID): void}, \texttt{+ dystans(lokacja: (double, double), przystanek: Przystanek): double}
              \item Opis: Zarządza danymi o postojach i pozwala na wyliczanie odległości do przystanku.
          \end{itemize}

    \item \textbf{SerwisAdministracyjny}: Zarządza informacjami o utrudnieniach i rozkładach.
          \begin{itemize}
              \item Metody: \texttt{+ pobierzUtrudnienia(): List<Utrudnienie>}, \texttt{+ dodajUtrudnienie(utrudnienie: Utrudnienie): void}, \texttt{+ uaktualnijRozkład(rozkład: List<Postoj>): void}, \texttt{+ dodajRozkład(rozkład: List<Postoj>): void}
              \item Opis: Zapewnia dostęp do danych o utrudnieniach drogowych oraz aktualizuje rozkłady jazdy.
          \end{itemize}

    \item \textbf{SerwisPasazera}: Udostępnia funkcjonalności dla pasażera, takie jak pobieranie map, rozkładów i obliczanie ETA.
          \begin{itemize}
              \item Metody: \texttt{+ obliczETA(przystanek: Przystanek): timedelta}, \texttt{+ pobierzMape(): Mapa}, \texttt{+ pobierzRozkład(idLinii: UUID): List<Postoj>}
          \end{itemize}

    \item \textbf{Mapa}: Klasa reprezentująca mapę z lokalizacjami pasażerów, pojazdów oraz postojów.
          \begin{itemize}
              \item Atrybuty: \texttt{- lokalizacjaPasazera: (double, double)}, \texttt{- lokalizacjePojazdow: Map<UUID, (double, double)>}, \texttt{- postoje: Map<idLinii, List<Postoj>>}
              \item Metody: \texttt{+ wyswietl(): void}
          \end{itemize}

    \item \textbf{KlientGPS}: Klient komunikujący się z modułem GPS, zwracający pozycję pojazdu lub użytkownika.
          \begin{itemize}
              \item Metody: \texttt{+ pobierzLokalizację(idPojazdu: UUID): (double, double)}, \texttt{+ pobierzLokalizację(idUzytkownika: UUID): (double, double)}
          \end{itemize}
\end{itemize}

\subsection{Relacje techniczne}

W diagramie systemowym uwzględniono:

\begin{itemize}
    \item \textbf{Asocjacje i zależności}:
          \begin{itemize}
              \item \texttt{KalkulatorETA} zależy od \texttt{SerwisPostojow} i \texttt{SerwisAdministracyjny} w celu pobrania niezbędnych danych do wyliczeń.
              \item \texttt{SerwisPasazera} korzysta z \texttt{KalkulatorETA} oraz \texttt{Mapa} do obsługi zapytań użytkownika.
              \item \texttt{KlientGPS} dostarcza dane lokalizacyjne dla \texttt{KalkulatorETA} oraz \texttt{Mapa}.
          \end{itemize}

    \item \textbf{Brak repozytoriów:}
          Na tym poziomie abstrakcji zdecydowano się nie modelować osobnych klas repozytoriów, gdyż serwisy pełnią funkcję warstwy dostępu do danych.
\end{itemize}

\subsection{Typy pomocnicze}

\begin{itemize}
    \item \textbf{Typy pomocnicze:}
          \begin{itemize}
              \item \texttt{UUID} – unikalny identyfikator obiektów takich jak pojazd, przystanek, postój itp.
              \item \texttt{timedelta} – typ reprezentujący odstęp czasu.
              \item \texttt{Mapa} – obiekt przechowujący aktualne lokalizacje i umożliwiający wizualizację.
          \end{itemize}
\end{itemize}



% --- PUNKT 8: ARCHITEKTURA I INTERFEJS ---
\newpage
\section{Architektura systemu i projekt interfejsu}

\subsection{Koncepcja architektury}
System został zaprojektowany w oparciu o \textbf{architekturę warstwową (n-tier architecture)}, co zapewnia separację logiki biznesowej od sposobu prezentacji danych.

\begin{itemize}
    \item \textbf{Warstwa Prezentacji (Frontend):} Aplikacja webowa i mobilna komunikująca się z serwerem poprzez API. Odpowiada za renderowanie mapy i tablic odjazdów.
    \item \textbf{Warstwa Logiki (Backend):} Centralny serwer przetwarzający dane. Tu znajduje się silnik ETA oraz moduły integracji z GPS.
    \item \textbf{Warstwa Danych (Database):} Relacyjna baza danych przechowująca rozkłady statyczne oraz nierelacyjna baza typu Key-Value dla szybkich aktualizacji pozycji pojazdów.
\end{itemize}

\subsection{Interfejsy komunikacyjne}
Komunikacja między modułami odbywa się za pomocą standardu \textbf{REST API}. Kluczowe punkty styku (Endpoints):
\begin{itemize}
    \item \texttt{GET /stops/\{id\}/departures} – pobiera listę odjazdów w czasie rzeczywistym.
    \item \texttt{POST /admin/alerts} – przesyła nowy komunikat o utrudnieniach.
    \item \texttt{GET /vehicles/positions} – pobiera współrzędne wszystkich pojazdów danej linii.
\end{itemize}

\subsection{Mocki widoków (User Interface)}
Poniższe makiety przedstawiają kluczowe ekrany aplikacji użytkownika końcowego.

\begin{figure}[h]
    \centering
    \includegraphics[width=0.6\textwidth]{widoki.png}
    \caption{Makiety interfejsu: Tablica odjazdów (lewo) oraz Mapa Live (prawo).}
\end{figure}

\textbf{Opis widoków:}
\begin{enumerate}
    \item \textbf{Widok Tablicy Przystankowej:} Wyświetla numer linii, kierunek oraz czas do odjazdu. Czas rzeczywisty jest wyróżniony kolorem zielonym lub ikoną fal radiowych.
    \item \textbf{Widok Mapy:} Interaktywna mapa z naniesioną pozycją użytkownika oraz ikonami autobusów/tramwajów poruszającymi się w czasie rzeczywistym.
\end{enumerate}
\subsection{Schemat blokowy architektury}
Poniższy diagram (Rysunek \ref{fig:architecture}) przedstawia fizyczny i logiczny podział systemu na komponenty oraz protokoły komunikacyjne użyte do ich integracji.

\begin{figure}[h]
    \centering
    \includegraphics[width=0.8\textwidth]{architektura.png}
    \caption{Schemat architektury warstwowej systemu TransTime.}
    \label{fig:architecture}
\end{figure}

\textbf{Komponenty infrastruktury:}
\begin{itemize}
    \item \textbf{Load Balancer:} Rozdziela ruch między instancje serwera API (zapewnienie wysokiej dostępności).
    \item \textbf{Redis Cache:} Przechowuje najświeższe dane o pozycjach GPS, aby nie obciążać głównej bazy danych przy każdym zapytaniu pasażera.
    \item \textbf{Worker Services:} Niezależne procesy w tle, które co kilka sekund odpytują Bazę danych i aktualizują estymacje czasowe.
\end{itemize}
\end{document}